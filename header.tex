
\section{Root element}
\label{sec:root-element}

All NAF documents have a root element \texttt{<NAF>} which has the following
attributes:

\begin{itemize}
\item \texttt{xml:lang} (\textbf{required}): language identifier .
\item \texttt{version} (\textbf{required}): the version of NAF. For
  newsreader, we will use version \textbf{v1}
\end{itemize}

Example:
\begin{Verbatim}[fontsize=\small]
<NAF xml:lang="en" version="v3">
  <!--- ... --->
</NAF>
\end{Verbatim}

\section{NAF header}
\label{sec:naf-header}

NAF documents may have a header for describing information about the
document, such as its original name, URI or a list of the linguistic
processors which generated the NAF document. The NAF header is represented
within the \texttt{<nafHeader>} element, which is optional but highly
recommended. The header element has the following sub-elements:

\subsection{fileDesc element}
\label{sec:filedesc-element}

\texttt{<fileDesc>} is an empty element containing information about the
original document itself. It has the following attributes:
\begin{itemize}
\item \texttt{title} (optional): the title of the document.
\item \texttt{author} (optional): the author of the document.
\item \texttt{creationtime} (optional): a timestamp following the
  \emph{xs:dateTime}\footnote{See
      \texttt{http://www.w3.org/TR/xmlschema-2/\#isoformats}. In
    summary, the date is specified following the form
    ``YYYY-MM-DDThh:mm:ss'' (all fields required). To specify a time
    zone, you can either enter a dateTime in UTC time by adding a "Z"
    behind the time (``2002-05-30T09:00:00Z'') or you can specify an
    offset from the UTC time by adding a positive or negative time
    behind the time (``2002-05-30T09:00:00+06:00'').} format that
  specifies the time when the document was created.
\item \texttt{filename} (optional): the original file name.
\item \texttt{filetype} (optional): the original format (PDF, HTML, DOC, etc).
\item \texttt{pages} (optional): number of pages of the original document.
\end{itemize}

Example:

\begin{Verbatim}[fontsize=\small]
 <fileDesc creationtime="2014-01-01T00:00:00Z"
           title="The best residence in the world."
           author="casa400"
           filename="residence_hostal"
           filetype="PDF" pages="19"/>
\end{Verbatim}

% \subsection{dcDesc element}
% \label{sec:dcdesc-element}

% \texttt{<dcDesc>} is an element containing sub-elements that mimic those of
% the Dublin Core \footnote{dublincore.org/documents/dcmi-terms/}. The
% elements which describe various aspects of the original
% document. Specifically, we include those sub-elements:

% \begin{itemize}
% % \item \texttt{<title>}: A title for the document.
% \item \texttt{<creator>}: An entity primarily responsible for making the document.
% \item \texttt{<subject>}: The topic of the document.
% \item \texttt{<description>}: An account of the document.
% \item \texttt{<publisher>}: An entity responsible for making the document available.
% \item \texttt{<contributor>}: An entity responsible for making contributions to the document.
% \item \texttt{<date>}: A point or period of time associated with an event
%   in the lifecycle of the resource. The content of the element should be of
%   \emph{xs:dateTyme} type (see below).
% \item \texttt{<type>}: The nature or genre of the resource.
% \item \texttt{<format>}: The file format, physical medium, or dimensions of the resource.
% %\item \texttt{dcidentifier}: ---
% % \item \texttt{<source>}: A related resource from which the described resource is derived.
% % \item \texttt{<language>}: A language of the resource.
% % \item \texttt{<relation>}: A related resource.
% % \item \texttt{<coverage>}: The spatial or temporal topic of the resource,
% %   the spatial applicability of the resource, or the jurisdiction under which
% %   the resource is relevant.
% \item \texttt{<rights>}: Information about rights held in and over the resource.
% \end{itemize}

% Please note:
% \begin{itemize}
% \item The contents of each of those sub-elements are textual. Only the
%   \texttt{<date>} elements has a \emph{xs:datetime} type, so that date
%   expressions are normalized. In \ref{sec:ling-proc} section we describe the
%   format of this type.
% \item There can be any number of those subelements inside a
%   \texttt{<dcDesc>} element.
% \end{itemize}

% Example:

% \begin{Verbatim}[fontsize=\small]
% <dcDesc>
%   <description>An example document for explaining NAF.</description>
%   <creator>Aitor</creator>
%   <subject>Example</subject>
%   <subject>Small</subject>
% </dcDesc>
% \end{Verbatim}


\subsection{public element}
\label{sec:public-element}

\texttt{<public>} is an empty element which stores public information about
the document, such as its URI. It has the following attributes:
\begin{itemize}
\item \texttt{publicId} (optional): a public identifier (for instance, the
  number inserted by the capture server, or the MD5 hash number of the
  original document).
\item \texttt{uri} (optional): a public URI of the document.
\end{itemize}

Example:
\begin{Verbatim}[fontsize=\small]
  <public publicId="50ee45d106f9caf2d1cf38f29419efa8"
          uri="http://casa400.com/docs/residence.pdf"/>
\end{Verbatim}

\subsection{Linguistic Processors}
\label{sec:ling-proc}

The header also stores the information about which linguistic processors
produced the NAF document, described under \texttt{<linguisticProcessors>}
elements. There can be several \texttt{<linguisticProcessors>} elements, one
per NAF layer. NAF layers correspond to the top-level elements of the
documents, such as "text", "terms", "deps" etc.  Each
\texttt{<linguisticProcessors>} element contains one or several
\texttt{<lp>} elements, each one describing one specific linguistic
processor.\\

The \texttt{<lp>} element, if present, has the following attributes:
\begin{itemize}
\item \texttt{name} (\textbf{required}): the name of the processor
\item \texttt{version} (optional): processor's version
\item \texttt{timestamp} (optional): a timestamp, denoting the
  date/time at which the processor was launched. It follows the XML
  Schema \emph{xs:dateTime} format.
\item \texttt{beginTimestamp} (optional): a timestamp, denoting the
  date/time at which the processor started the process. It follows the XML Schema
  \emph{xs:dateTime} format.
\item \texttt{endTimestamp} (optional): a timestamp, denoting the date/time
  at which the processor ended the process. It follows the XML Schema
  \emph{xs:dateTime} format.
\item \texttt{hostname} (optional): The name of the machine where the
  processor was ran..
\end{itemize}

Example:

\begin{Verbatim}
<linguisticProcessors layer="text">
   <lp name="Freeling" version="2.1"
       timestamp="2012-06-25T10:05:00Z"
       beginTimestamp="2012-06-25T10:05:00Z"
       endTimestamp="2012-06-25T10:12:00Z"/>
 </linguisticProcessors>
 <linguisticProcessors layer="terms">
  <lp name="Freeling" version="2.1"
      timestamp="2012-06-25T10:10:19Z"
      beginTimestamp="2012-06-25T10:10:19Z"
      endTimestamp="2012-06-25T10:15:19Z"/>
  <lp name="ukb" version="0.1.2"
      timestamp="2012-06-25T16:10:19Z"
      beginTimestamp="2012-06-25T16:10:19Z"
      endTimestamp="2012-06-25T16:20:19Z"/>
</linguisticProcessors>
<linguisticProcessors layer="namedEntities">
  <lp name="Standfort_NE"
      version="0.1"
      timestamp="20090626_00:10:19Z"
      beginTimestamp="20090626_00:10:19Z"
      endTimestamp="20090626_00:14:19Z"/>
</linguisticProcessors>
\end{Verbatim}

Here is a full example of a NAF header:

\begin{Verbatim}
<nafHeader>
 <fileDesc creationtime="2014-01-01T00:00:00Z"
           title="The best residence in the world."
           author="casa400"
           filename="residence_hostal"
           filetype="PDF" pages="19"/>
  <public publicId="3_3012"
          uri="http://casa400.com/docs/residence.pdf" />
  <linguisticProcessors layer="text">
     <lp name="Freeling" version="2.1"
         timestamp="2012-06-25T10:05:00Z"
         beginTimestamp="2012-06-25T10:05:00Z"
         endTimestamp="2012-06-25T10:12:00Z"/>
   </linguisticProcessors>
   <linguisticProcessors layer="terms">
    <lp name="Freeling" version="2.1"
        timestamp="2012-06-25T10:10:19Z"
        beginTimestamp="2012-06-25T10:10:19Z"
        endTimestamp="2012-06-25T10:15:19Z"/>
    <lp name="ukb" version="0.1.2"
        timestamp="2012-06-25T16:10:19Z"
        beginTimestamp="2012-06-25T16:10:19Z"
        endTimestamp="2012-06-25T16:20:19Z"/>
  </linguisticProcessors>
  <linguisticProcessors layer="namedEntities">
    <lp name="Standfort_NE"
        version="0.1"
        timestamp="2009-06-26T00:10:19Z"
        beginTimestamp="2009-06-26T00:10:19Z"
        endTimestamp="2009-06-26T00:14:19Z"/>
  </linguisticProcessors>
</nafHeader>
\end{Verbatim}


%%% Local Variables: 
%%% mode: latex
%%% TeX-master: "naf"
%%% End: 
