\section{Attribution Layer}\label{sec:attribution}

The attribution layer is meant to represent information about the source of a statement, when this is explicitly mentioned in the text. It consists of {\tt <statement>} elements that contain, at least, the  {\tt <statement\_target>} and, typically, the {\tt <statement_source>}. If present, the  {\tt <statement\_cue>} can also be represented. All elements contain a span that points to relevant terms in the term layer.

A  {\tt <statement>} has the following subelements:

\begin{itemize}
\item statement\_target (\textbf{required}): provides the span of terms that constitute the statement itself, i.e.\ where the attribution applies to (in the example below \textit{Hilton Hotel Paris was a nightmare}).
\item statement\_source: provides the span that expresses the source of the statement (\textit{They} in the example)
\item statement\_cue: provides the span that explicitly links the statement to the source (\textit{said} in the example, but also expressions such as \textit{according to})
\end{itemize}  


\begin{verbatim}
<attribution>
    <!-- They said Hilton Hotel Paris was a nightmare. -->
    <statement id="a1">
        <statement_target>
            <span>
                <target id="t3">
                <target id="t4">
                <target id="t5">
                <target id="t6">
                <target id="t7">
            </span>
        </statement_target>
        <statement_source>
            <span>
                <target id="t1">
            </span>
        </statement_source>
        <statement_cue>
            <span>
                <target id="t2">
            </span>
        </statement_cue>
    </statement>
</attribution>
\end{verbatim}

%%% Local Variables: 
%%% mode: latex
%%% TeX-master: "naf"
%%% End: 
