
\section{Terms}
\label{sec:terms}

Terms are lexical units which refer to word forms (and groups multi word
forms). Terms are attached to several morphosyntactic information (such as
part of speech, lemma, etc), and may be linked to external resources like
WordNet senses, etc.\\

The \texttt{<term>} element has the following attributes:
\begin{itemize}
\item \texttt{id} (\textbf{required}): unique identifier starting with the
  prefix ``t''. If the term is a compound term (see Section
  \ref{sec:compound-terms}), the prefix is ``t.mw''.
\item \texttt{type} (optional): type of the term. Currently, 2 values are
  possible:
  \begin{itemize}
  \item \texttt{open}: open category term
  \item \texttt{close}: close category term
  \end{itemize}
\item \texttt{lemma} (optional): lemma of the term
\item \texttt{pos} (optional): part of speech. The first letter of the pos
  attribute must be one of the following:
  \begin{itemize}
  \item [N] common noun
  \item [R] proper noun
  \item [G] adjective
  \item [V] verb
  \item [P] preposition
  \item [A] adverb
  \item [C] conjunction
  \item [D] determiner
  \item [O] other
  \end{itemize}
\item \texttt{morphofeat} (optional): morphosyntactic feature encoded as a
  single attribute.
\item \texttt{head} (optional): if the term is a compound, the id of the
  head component (see Section \ref{sec:compound-terms}).
\item \texttt{case} (optional): declension case of the term.
\end{itemize}

The \texttt{<term>} element have the following sub-element:
\begin{itemize}
\item \texttt{span}: used to identify the tokens that the term spans. A
  single term may refer to one or more words, in case of multiword
  expressions. The \texttt{<span>} element has as many \texttt{<target>}
  sub-elements as spanned tokens (see example below).
\end{itemize}

Example of term level annotations:
\begin{verbatim}[fontsize=\small]
<terms>
  <term id="t1" lemma="John" pos="R">
    <span>
      <target id="w1"/>
    </span>
  </term>
  <term id="t2" type="open" lemma="teach" pos="V">
    <span>
      <target id="w2"/>
    </span>
  </term>
  <term id="t3" lemma="mathematics" pos="N">
    <span>
      <target id="w3"/>
    </span>
  </term>
  <term id="t4" lemma="20" pos="N">
    <span>
      <target id="w4"/>
    </span>
  </term>
  <term id="t5" lemma="minute" pos="N">
    <span>
      <target id="w5"/>
    </span>
  </term>
  <term id="t5" lemma="every" pos="D">
    <span>
      <target id="w6"/>
    </span>
  </term>
  <term id="t6" lemma="Monday" pos="N">
    <span>
      <target id="w7"/>
    </span>
  </term>
  <term id="t7" lemma="in" pos="P">
    <span>
      <target id="w8"/>
    </span>
  </term>
  <term id="t.mw8" lemma="New_York" pos="R">
    <span>
      <target id="w9"/>
      <target id="w10"/>
    </span>
  </term>
</terms>
\end{verbatim}


\subsection{Sentiment features}
\label{sec:sentiment-features}

The term layer represents sentiment information which is context-independent
and that can be found in a sentiment lexicon. It is related to concepts
expressed by words/ terms (e.g. ``beautiful'') or multi-word expressions
(e. g. ``out of order''). NAF provides possibilities to store sentiment
information at term level and at sense/synset level. In the latter case, the
sentiment information is included in the \texttt{<externalReference>}
section (Section \ref{sec:external-references}) and a word sense
disambiguation process may identify the correct sense along its sentiment
information. \\

%The extension contains the following information categories.

The \texttt{<sentiment>} element has the following attributes:
\begin{itemize}
\item \texttt{Resource} (\textbf{required}): identifier and reference to an external sentiment
  resource.
\item \texttt{Polarity} (\textbf{required}): Refers to the property of a
  word/sense to express positive, negative or no sentiment. These values are
  possible:
  \begin{itemize}
  \item positive
  \item negative
  \item neutral
  \item Or a numerical value.
  \end{itemize}
\item \texttt{Strength} (optional): refers to the strength of the polarity.
  \begin{itemize}
  \item weak
  \item average
  \item strong
  \item Or a numerical value
  \end{itemize}
\item \texttt{Subjectivity} (optional): refers to the property of a words to express an
  opinion (or not).
  \begin{itemize}
  \item subjective
  \item objective
  \item factual
  \item opinionated
  \end{itemize}
\item \texttt{Sentiment\_semantic\_type} (optional): refers to a
  sentiment-related semantic type. Possible values:
  \begin{itemize}
  \item Aesthetics\_evaluation
  \item Moral\_judgment
  \item Emotion
  \item etc.
  \end{itemize}
\item \texttt{Sentiment\_modifier} (optional): refers to words which modify
  the polarity of another word. Possible values:
  \begin{itemize}
  \item intensifier polarity shifter
  \item weakener polarity shifter
  \end{itemize}
\item \texttt{Sentiment\_marker} (optional): refers to words which
  themselves do not carry polarity, but are kind of vehicles of it. Possible
  values:
  \begin{itemize}
  \item Find, think, in my opinion, according to ...
  \end{itemize}
\item \texttt{Sentiment\_product\_feature} (optional): refers to a domain;
  mainly used in feature-based sentiment analysis. Values are related to
  specific domain. These are some possible values for for the tourist
  domain:
  \begin{itemize}
  \item staff, cleanliness, beds, bathroom, transportation, location, etc..
  \end{itemize}
\end{itemize}

Table \ref{tab:sent-attr-some} shows sentiment values for some words.

\begin{table}[t]
  \centering
  \ttfamily
  \small
  \begin{tabular}{|c|l|}
    \hline
    \textrm{Word} & \textrm{Sentiment attributes}\\
    \hline
    \multirow{6}{*}{\textrm{beatiful}} & polarity = "positive" \\
    & strength = "average" \\
    & subjectivity = "subjective"\\
    & sentiment\_semantic\_type = "aesthetics\_evaluation"\\
    & sentiment\_modifier = ""\\
    & sentiment\_product\_feature = "general"\\
    \hline
    \multirow{6}{*}{\textrm{valley}} & polarity = "negative"\\
    & strength = "average"\\
    & subjectivity = "factual"\\
    & sentiment\_semantic\_type = ""\\
    & sentiment\_modifier = ""\\
    & sentiment\_product\_feature = "beds"\\
    \hline
    \multirow{6}{*}{\textrm{very}} & polarity = ""\\
    & strength = ""\\
    & subjectivity = ""\\
    & sentiment\_semantic\_type = ""\\
    & sentiment\_modifier = "intensifier"\\
    & sentiment\_product\_feature = ""\\
    \hline
  \end{tabular}
  \caption{\textrm{Sentiment attributes.}}
  \label{tab:sent-attr-some}
\end{table}


Example:
\begin{verbatim}[fontsize=\small]
<!-- term level sentiment annotation -->
<term id="t2" lemma="nice" pos="G">
  <sentiment resource="VUA_polarityLexicon_word" polarity="positive"
    strength="average" subjectivity="subjective"
    sentiment_semantic_type="behaviour/trait" />
  <span><target id="w2"/></span>
<!-- sense level sentiment annotation -->
</term>
<term id="t5" lemma="warm" pos="G">
  <sentiment/>
  <span><target id="w5"/></span>
  <externalReferences>
    <externalRef resource="WN-ENG" reference="c_1009" conf="0.38">
      <sentiment resource="VUA_polarityLexicon_synset" polarity="positive"
          strength="average" subjectivity="subjective"
         sentiment_semantic_type="behaviour/traitEvaluation"/>
    </externalRef>
    <externalRef resource="WN-ENG" reference="c_1008" conf="0.31">
      <sentiment resource="VUA_polarityLexicon_synset" polarity="positive"
         strength="average" subjectivity="objective"
         sentiment_semantic_type="temperature"/>
    </externalRef>
  </externalReferences>
</term>
\end{verbatim}

\subsection{External References}
\label{sec:external-references}

The \texttt{<externalReferences>} element is used to associate terms to
external resources, such as elements of a Knowledge base, an ontology,
etc. It consists of several \texttt{<externalRef>} elements, one per
association. The \texttt{<externalRef>} elements may be nested, meaning that
each externalRef refines the relation expressed by the parent element.\\

The \texttt{<externalRef>} element has the following attributes:
\begin{itemize}
\item \texttt{resource} (\textbf{required}): indicates the identifier of the
  resource referred to.
\item \texttt{reference} (\textbf{required}): code of the referred
  element. If the element is a synset of some version of WordNet, it is
  recommended to follow the pattern:\\
  \begin{center}
    \texttt{[a-z]{3}-[0-9]{2}-[0-9]+-[nvars]}
  \end{center}
  which is a string composed by four fields separated by a dash. The four
  fields are the following:
  \begin{itemize}
  \item Language code (three characters, lowercase).
  \item WordNet version (two digits).
  \item Synset identifier composed by digits.
  \item POS character:
    \begin{itemize}
    \item [n] noun
    \item [v] verb
    \item [a] adjective
    \item [r] adverb
    \end{itemize}
  \end{itemize}
  examples of valid patterns are: ``eng-20-12345678-n'',
  ``spa-16-017403-v'', etc.
\item \texttt{reftype} (optional): indicates the kind of relation the
  externalRef is expressing.
\item \texttt{status} (optional): indicates the status of the relationship.
\item \texttt{source} (optional): the name of the process which created the
  external reference.
\item \texttt{confidence} (optional): a floating number between 0 and
  1. Indicates the confidence weight of the association.
\end{itemize}

Example of term level annotations:
\begin{verbatim}[fontsize=\small]
<terms>
  <term id="t1" lemma="John" pos="R">
    <span>
      <target id="w1"/>
    </span>
  </term>
  <term id="t2" type="open" lemma="teach" pos="V">
    <span>
      <target id="w2"/>
    </span>
    <externalReferences>
      <externalRef resource="WN-1.7" reference="eng-17-00861095-v"
                   confidence="0.80">
        <externalRef resource="ontology" reference="Teach"
                     reftype="SubClassOf">
        </externalRef>
      </externalRef>
      <externalRef resource="WN-1.7" reference="eng-17-00859568-v"
                   confidence="0.20"/>
    </externalReferences>
  </term>
  <term id="t3" lemma="mathematics" pos="N">
    <span>
      <target id="w3"/>
    </span>
    <externalReferences>
      <externalRef resource="WN-1.7" reference="eng-17-04597590-n"
                   confidence="1.0"/>
    </externalReferences>
  </term>
  <term id="t4" lemma="20" pos="N">
    <span>
      <target id="w4"/>
    </span>
  </term>
  <term id="t5" lemma="minute" pos="N">
    <span>
      <target id="w5"/>
    </span>
  </term>
  <externalReferences>
    <externalRef resource="WN-1.7" reference="eng-17-12621100-n"
                 confidence="0.80"/>
    <externalRef resource="WN-1.7" reference="eng-17-12631889-n"
                 confidence="0.20"/>
  </externalReferences>
  <term id="t5" lemma="every" pos="D">
    <span>
      <target id="w6"/>
    </span>
  </term>
  <term id="t6" lemma="Monday" pos="N">
    <span>
      <target id="w7"/>
    </span>
    <externalReferences>
      <externalRef resource="WN-1.7" reference="eng-17-12557842-n"
                   confidence="1.0"/>
    </externalReferences>
  </term>
  <term id="t7" lemma="in" pos="P">
    <span>
      <target id="w8"/>
    </span>
  </term>
  <term id="mw8" lemma="New_York" pos="R">
    <span>
      <target id="w9"/>
      <target id="w10"/>
    </span>
  </term>
</terms>
\end{verbatim}

\subsection{Compound terms}
\label{sec:compound-terms}

Compound terms can be represented in NAF by including \texttt{<component>}
elements within \texttt{<term>} elements. For example, the term
landbouwbeleid (English: agriculture policy) would look like this:

\begin{verbatim}[fontsize=\small]
<term id="t.mw7" head="t.mw7.1" lemma="landbouwbeleid" pos="N" type="open">
  <span>
    <target id="w7"/>
  </span>
  <component id="t.mw7.1" lemma="landbouw" pos="N">
    <externalReferences>...</externalReferences>
  </component>
  <component id="t.mw7.2" lemma="beleid" pos="N">
    <externalReferences>...</externalReferences>
  </component>
  <externalReferences>...</externalReferences>
</term>
\end{verbatim}

Note: the \texttt{id} attribute of compound terms start with the prefix ``t.mw''.\\

The \texttt{<component>} element has the following attributes:
\begin{itemize}
\item \texttt{id} (\textbf{required)}: unique identifier for the component
  starting with the same prefix as its parent \texttt{<term>} element. For
  instance, if the parent \texttt{<term>} element's \texttt{id} is
  \texttt{t.mw7}, the prefix for the \texttt{id} attribute of the
  \texttt{<component>} elements is \texttt{t.mw7} (for instance,
  \texttt{t.mw7.1}, \texttt{t.mw7.2}, etc).
\item \texttt{lemma} (optional): lemma of the term.
\item \texttt{head} (optional): the identifier of the head component (one of
  the \texttt{<component>} sub-element).
\item \texttt{pos} (optional): part of speech.
\item \texttt{case} (optional): declension case.
\end{itemize}


%%% Local Variables: 
%%% mode: latex
%%% TeX-master: "naf"
%%% End: 
