\section{Factuality}
\label{sec:factuality}

The factuality layer encodes the veracity or \emph{factuality} of events as
mentioned in the text. This information is useful for recognizing whether
the events mentioned in the text actually happened (factual events), did not
happen (contrafactual events), or there is some uncertainty about the event
occurring or not.

The factuality information is described under the \texttt{<factualityLayer>}
element. Each piece of information regarding factuality of an event is
represented by an empty element \texttt{<factvalue>}, whose attributes are
the following:

\begin{itemize}
\item \texttt{id} (\textbf{required}): the word id of the event token
\item \texttt{prediction} (\textbf{required}): the factuality value (nominal
  value) that is predicted by the classifier, according to the
  FactBank~\cite{sauri2009factbank} factuality classification scheme.
\item \texttt{confidence} (optional): a numerical value indicating the
  confidence the classifier has in its prediction.
\end{itemize}

\begin{verbatim}
<factualitylayer>
  <factvalue id="w179" prediction="CT+" confidence="0.89"/>
  <factvalue id="w108" prediction="CT+" confidence="0.93"/>
</factualitylayer>
\end{verbatim}

%%% Local Variables: 
%%% mode: latex
%%% TeX-master: "naf"
%%% End: 

