
\section{Semantic Role Labeling}
\label{sec:semant-role-label}

Semantic Role Labeling (SRL) is a shallow semantic analysis which detects
semantic arguments associated with predicates. In NAF, the SRL information
is stored on a separate layer, whose top element is \texttt{<srl>}.

Each annotated predicate is represented by an \texttt{<predicate>}
element. The \texttt{<predicate>} element has the following attributes:
\begin{itemize}
\item \texttt{id} (\textbf{required}): the identifier of the predicate,
  starting with the prefix ``pr''.
\item \texttt{uri} (optional): An URI to an external resource representing the model of
  an event/predicate. For instance, it could be a reference to a frame in
  \emph{FrameNet} \cite{Baker+'98}, \texttt{PropBank} \cite{propbank02}, etc.
\item \texttt{confidence} (optional): a floating number between 0 and
  1. Indicates the confidence weight of the association.
\end{itemize}


Inside \texttt{<predicate>}, there are the following sub-elements:
\begin{itemize}
\item One \texttt{<externalReferences>} element, which links the predicate
  with one or more external resources.
\item One \texttt{<span>} element, spaning terms, and which point to the
  surface form of the predicate in the document.
\item One or more \texttt{<role>} elements which fill the arguments of the
  predicate.
\end{itemize}

As always in NAF, the \texttt{<span>} element contains one or more
\texttt{<target>} elements with the following attribute:

\begin{itemize}
\item \texttt{id} (\textbf{required}): id of the target term.
\end{itemize}

The \texttt{<role>} element represents filler of a particular argument of
the predicate. It has the following attributes:
\begin{itemize}
\item \texttt{id} (\textbf{required}): the id of the role, starting with
  prefix ``rl''.
\item \texttt{semRole} (\textbf{required}): the role type (for instance,
  ``A0'' or ``A1'' if the role belongs to a PropBank predicate).
\item \texttt{uri} (\textbf{optional}): an URI pointing to an externar
  reference representing the particular role.
\item \texttt{confidence} (\textbf{optional}): a confidence value.
\end{itemize}

Inside \texttt{<role>}, there are the following sub-elements:
\begin{itemize}
\item One \texttt{<externalReferences>} element, which links the predicate
  with one or more external resources.
\item One \texttt{<span>} element, spaning terms, and which point to the
  surface form of the predicate in the document.
\end{itemize}

Each \texttt{<span>} contains one or more \texttt{<target>}
elements, with the following attributes:

\begin{itemize}
\item \texttt{id} (\textbf{required}): id of the target term.
\end{itemize}

Given the following sentence:

\begin{quote}
  She was making apple jam for her friends and it was a surprise.
\end{quote}

The SRL layer would be encoded as follows:

\begin{Verbatim}[fontsize=\small]
<srl>
  <predicate id="pr1" uri="http://framenet.net/Make" confidence=0.9>
    <!-- making -->
     <externalReferences>
        <externalRef reference="cognition" reftype="mcr:cognition" resource="2"/>
        <externalRef reference="cognition" reftype="mcr:cognition" resource="mc"/>
        <externalRef reference="eng-30-00690614-v" resource="wn:eng-30"/>
      </externalReferences>
    <span>
      <target id="t3"/>
    </span>
    <role id="rl1" semRole="Agent">
      <!-- She -->
      <externalReferences>
        <externalRef reference="participant" resource="1"/>
      </externalReferences>
      <span>
	<target id="t1"/>
      </span>
    </role>
    <role id="rl2" semRole="Theme">
      <!-- apple jam -->
      <span>
	<target id="t4"/>
	<target id="t5"/>
      </span>
    </role>
  </predicate>
</srl>
\end{Verbatim}


%%% Local Variables: 
%%% mode: latex
%%% TeX-master: "naf"
%%% End: 
