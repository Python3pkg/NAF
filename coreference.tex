
\section{Coreference}
\label{sec:coreference}

The coreference layer creates clusters of term spans (which we call
mentions) which share the same referent. For instance, ``London'' and ``the
capital city of England'' are two mentions referring to the same entity. It
is said that those mentions corefer.  A \texttt{<coref>} element represents
a mention cluster, and within \texttt{<coref>} each mention is represented
by a \texttt{<span>} element (which groups term mentions using
\texttt{<target>} elements). Additionally, one \texttt{<target>} element
within the \texttt{<span>} may have an attribute \texttt{head} with value
``yes'' to represent the fact that this particular term is the head of the
mention. For instance, the head of the mention ``the capital city of
England'' is ``city''.\\

The \texttt{<coref>} element has the following attribute:
\begin{itemize}
\item \texttt{id} (\textbf{required}): unique id, starting with the prefix ``co''.
\item \texttt{type} (optional): type of the coreference set. It describes
  whether the coreference cluster refers to an entity instance, and event,
  etc.
\end{itemize}

The \texttt{<coref>} element contains as many \texttt{<span>} elements as
elements in the coreference cluster. Each \texttt{<span>} contains one or
more \texttt{<target>} elements, with the following attributes:
\begin{itemize}
\item \texttt{id} (\textbf{required}): id that refers to the target term.
\item \texttt{head} (optional): a "yes" value indicates that the term is the
  head of the coreference cluster.
\end{itemize}

Example of a coreference cluster:

\begin{verbatim}[fontsize=\small]
<coreferences>
  <coref id="co1" type="entity">
    <!-- London -->
    <span >
      <target id="t12" head="yes"/>
    </span>
    <!-- the capital city of England -->
    <span>
      <target id="t1"/>
      <target id="t2"/>
      <target id="t3" head ="yes"/> <!-- city is the head -->
      <target id="t4"/>
      <target id="t5"/>
    </span>
  </coref>
</coreferences>
\end{verbatim}


%%% Local Variables: 
%%% mode: latex
%%% TeX-master: "naf"
%%% End: 
