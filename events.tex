
\section{Events}
\label{sec:events}

% @@TODO: intro about events\\

\subsection{Time expressions}
\label{sec:time-expressions}

NAF follows TimeML schema for annotating time expressions under the
\texttt{timeExpressions} layer. Inside \texttt{timeExpressions} there are
many \texttt{<timex3>} elements, each one describing one time
expression. The \texttt{<timex3>} in NAF have the same semantics as
TimeML. The only difference is that TimeML inserts \texttt{<timex3>}
elements along with the running text in the document, wheras NAF uses a
stand-off approach. Please refer to TimeML specification document
\cite{isotimeml} for further information.

The \texttt{<timex3>} element has the following attributes:
\begin{itemize}
\item \texttt{id} (\textbf{required}): unique identifier starting with
  the prefix ``tmx'' (``tmx1'', ``tmx2'', etc.)

\item \texttt{type} (\textbf{required}): type of the timex3
  expression. Possible values:
  \begin{itemize}
  \item DATE
  \item TIME
  \item DURATION
  \item SET
  \end{itemize}

\item \texttt{beginPoint} (optional): term id of beginning point.
\item \texttt{endPoint} (optional): term id of end point.
\item \texttt{quant} (optional): used for specifying sets that denote
  quantified times. Generally a literal from the text that quelifies over
  the expression. Usual values are ``EVERY'', ``SOME'', etc.
\item \texttt{freq} (optional): Used for specifying sets that denote
  quantified times. It contains an integer value and a time granularity to
  represent any frequency contained in the set. Usual values are ``2X''
  (twice-a-month), ``3D'' (three-days), etc.

\item \texttt{functionInDocument} (default value: ``NONE''): indicates the
  function of the timex3 providing an anchor to other temporal expressions
  in the document. One of this values:

  \begin{tabular}{|c|p{9cm}|}
    \hline
    Value & Description \\
    \hline
    CREATION\_TIME     & date and time the identified resource where first created.\\
    EXPIRATION\_TIME   & date and time when the right to publish material expires.\\
    MODIFICATION\_TIME & date and time the resource was last modified.\\
    PUBLICATION\_TIME  & date and time when the resource is released to the public.\\
    RELEASE\_TIME      & date and time when the resource may be distributed.\\
    RECEPTION\_TIME    & date and time when the resource was received on current system.\\
    NONE & ---\\
    \hline

  \end{tabular}

\item \texttt{temporalFunction} (default: ``false''): whether the timex3 is
  used as a temporal function (``two weeks ago''). Possible values are
  ``true'' or ``false''.

\item \texttt{value} (optional): XML datatype based on 2002 TIDES guideline
  (expanding ISO 8601) for representing dates, times and durations. Possible
  values are ``Duration'', ``Time'', ``WeekDate'', ``Season'',
  ``PartOfYear'', ``ParPrFu'', etc.

\item \texttt{valueFromFunction} (optional): used when the value is taken
  from a temporal function timex3. The value is the identifier (tmx3id) of
  the timex3.

\item \texttt{mod} (optional): Used for temporal modifiers that cannot be
  expressed either within value proper, or via links or temporal
  functions. Possible values:
  \begin{itemize}
  \item BEFORE
  \item AFTER
  \item ON\_OR\_BEFORE
  \item ON\_OR\_AFTER
  \item LESS\_THAN
  \item MORE\_THAN
  \item EQUAL\_OR\_LESS
  \item EQUAL\_OR\_MORE
  \item START
  \item MID
  \item END
  \item APPROX
  \end{itemize}

\item \texttt{anchorTimeID} (optional): point to another timex3 in case of
  expression such as ``last week'', which have a functional
  interpretation. \texttt{anchorTimeID} provides the reference point to
  which the function interpretation applies. The value is the identifier
  (tmx3id) of the timex3.
\item \texttt{comment} (optional): optional comment.
\end{itemize}

The \texttt{<timex3>} element has one sub-element, \texttt{<span>}, which
groups the word forms the expression is referring to\footnote{Note that
  usually span elements refer to terms, but timex3 span elements reger to
  word forms.}.

\begin{Verbatim}[fontsize=\small]
<timeExpressions>
  <timex3 id="tmx1" type="DURATION">
    <span>
      <target id="w17"/>
      <target id="w18"/>
      <target id="w19"/>
      <target id="w20"/>
    </span>
  </timex3>
  <timex3 id="tmx2" type="DATE">
    <span>
      <target id="w22"/>
    </span>
  </timex3>
</timeExpressions>
\end{Verbatim}

%%% Local Variables:
%%% mode: latex
%%% TeX-master: "naf"
%%% End:
