
\section{Chunks}
\label{sec:chunks}

Chunks are noun or prepositional phrases, spanning terms.\\

The \texttt{<chunk>} element has the following attributes:
\begin{itemize}
\item \texttt{id} (\textbf{required}): unique identifier, starting with the prefix ``c''.
\item \texttt{head} (\textbf{required)}: the id of the chunk's head.
\item \texttt{phrase} (optional): type of the phrase.
\item \texttt{case} (optional): declension case.
\end{itemize}

Example of chunk annotations:
\begin{Verbatim}[fontsize=\small]
<chunks>
  <!-- John -->
  <chunk id="c1" head="t1" phrase="NP">
    <span>
      <target id="t1"/>
    </span>
  </chunk>
  <!-- taught -->
  <chunk id="c2" head="t2" phrase="V">
    <span>
      <target id="t2"/>
    </span>
  </chunk>
  <!-- Mathematics -->
  <chunk id="c3" head="t3" phrase="NP">
    <span>
      <target id="t3"/>
    </span>
  </chunk>
  <!-- 20 minutes -->
  <chunk id="c5" head="t5" phrase="NP">
    <span>
      <target id="t4"/>
      <target id="t5"/>
    </span>
  </chunk>
  <!-- every -->
  <chunk id="c6" head="t6" phrase="R">
    <span>
      <target id="t6"/>
    </span>
  </chunk>
  <!-- every Monday -->
  <chunk id="c7" head="t7" phrase="NP">
    <span>
      <target id="t6"/>
      <target id="t7"/>
    </span>
  </chunk>
  <!-- in New York -->
  <chunk id="c9" head="t9" phrase="PP">
    <span>
      <target id="t8"/>
      <target id="t9"/>
    </span>
  </chunk>
</chunks>
\end{Verbatim}


%%% Local Variables: 
%%% mode: latex
%%% TeX-master: "naf"
%%% End: 
