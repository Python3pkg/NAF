
\section{Topics layer}
\label{sec:topics-layer}

This layer encodes the topics for the document. The topics correspond
to the whole document, and are usually assigned by automatic methods,
such as text classification processors. Note that topics defined in the original metadata of the document will be represented in the \texttt{<metadata_info} layer. 
The topics are annotated within the \texttt{<topics>} element, and each topic is enclosed by a
\texttt{<topic>} element.

The \texttt{<topic>} element has the following attributes:
\begin{itemize}
\item \texttt{source} (optional): A reference to the entity
  responsible for creating the annotation.
\item \texttt{method} (optional): The name of the method used to
  create the annotation. This attribute is usually used in conjunction
  with the \texttt{source} attribute.
\item \texttt{confidence} (optional): this attribute is optional but
  its presence is highly recommended. It gives a value of
  ``confidence'' for the annotation. The confidence value can be in
  fact almost anything (similarity score, the value of the margin on a
  SVM based classification, etc), as long as it can be used to sort
  all annotations sharing the \texttt{source} and \texttt{method}
  attributes.
\item \texttt{uri} (optional): if the topic is a resource
  from an external reference, an URI to this resource. For instace, it
  could be a URI pointing to a Wikipedia category page, etc.
\end{itemize}

The content of the \texttt{<topic>} element is a string with the
topic.

Here is an example of topic annotation:

\begin{Verbatim}
<topics>
  <topic source="newsreader"
	 confidence="0.8"
	 method="textclassification-svm"
	 URI="http://en.wikipedia.org/wiki/Category:Tower_mills">
   Tower Mills</topic>
  <topic confidence="0.2"
	 source="newsreader"
	 method="textclassification-svm"
	 URI="http://en.wikipedia.org/wiki/Category:Windmills_in_Somerset">
   Windmills in Somerset</topic>
</topics>
\end{Verbatim}


%%% Local Variables: 
%%% mode: latex
%%% TeX-master: "naf"
%%% End: 
