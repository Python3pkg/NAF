
\section{Time expressions and events}
\label{sec:time-expr-events}

This section describes representing relations among events, events and time
expressions, or relations among time expressions. We first describe how to
represent events and time relations. Then, we describe how to represent
relations among them.

\subsection{Events}
\label{sec:events}

Within NAF, events are represented as coreference clusters in the
coreference layer \ref{sec:coreference}. Thus, the way to refer to those
events is to just span to a coreference element (c.f. Section
\ref{sec:coreference}). For instance, the following example:

\begin{Verbatim}
<coreferences>
  <coref id="coevent2" type="event">
    <span>
      <!--liked-->
      <target id="t13"/>
    </span>
  </coref>
  <coref id="coevent1" type="event">
    <span>
      <!--taught-->
      <target id="t2"/>
    </span>
  </coref>
</coreferences>
\end{Verbatim}

\noindent the coreference clusters \texttt{coevent2} and
\texttt{coevent1} (of type ``event'') represent two events.

\subsection{Time expressions}
\label{sec:time-expressions}

NAF follows TimeML schema for annotating time expressions under the
\texttt{timeExpressions} layer. Inside \texttt{timeExpressions} there are
many \texttt{<timex3>} elements, each one describing one time
expression. The \texttt{<timex3>} in NAF have the same semantics as
TimeML. The only difference is that TimeML inserts \texttt{<timex3>}
elements along with the running text in the document, wheras NAF uses a
stand-off approach. Please refer to TimeML specification document
\cite{isotimeml} for further information.

The \texttt{<timex3>} element has the following attributes:
\begin{itemize}
\item \texttt{id} (\textbf{required}): unique identifier starting with
  the prefix ``tmx'' (``tmx1'', ``tmx2'', etc.)

\item \texttt{type} (\textbf{required}): type of the timex3
  expression. Possible values:
  \begin{itemize}
  \item DATE
  \item TIME
  \item DURATION
  \item SET
  \end{itemize}

\item \texttt{beginPoint} (optional): term id of beginning point.
\item \texttt{endPoint} (optional): term id of end point.
\item \texttt{quant} (optional): used for specifying sets that denote
  quantified times. Generally a literal from the text that quelifies over
  the expression. Usual values are ``EVERY'', ``SOME'', etc.
\item \texttt{freq} (optional): Used for specifying sets that denote
  quantified times. It contains an integer value and a time granularity to
  represent any frequency contained in the set. Usual values are ``2X''
  (twice-a-month), ``3D'' (three-days), etc.

\item \texttt{functionInDocument} (default value: ``NONE''): indicates the
  function of the timex3 providing an anchor to other temporal expressions
  in the document. One of this values:

  \begin{tabular}{|c|p{9cm}|}
    \hline
    Value & Description \\
    \hline
    CREATION\_TIME     & date and time the identified resource where first created.\\
    EXPIRATION\_TIME   & date and time when the right to publish material expires.\\
    MODIFICATION\_TIME & date and time the resource was last modified.\\
    PUBLICATION\_TIME  & date and time when the resource is released to the public.\\
    RELEASE\_TIME      & date and time when the resource may be distributed.\\
    RECEPTION\_TIME    & date and time when the resource was received on current system.\\
    NONE & ---\\
    \hline

  \end{tabular}

\item \texttt{temporalFunction} (default: ``false''): whether the timex3 is
  used as a temporal function (``two weeks ago''). Possible values are
  ``true'' or ``false''.

\item \texttt{value} (optional): XML datatype based on 2002 TIDES guideline
  (expanding ISO 8601) for representing dates, times and durations. Possible
  values are ``Duration'', ``Time'', ``WeekDate'', ``Season'',
  ``PartOfYear'', ``ParPrFu'', etc.

\item \texttt{valueFromFunction} (optional): used when the value is taken
  from a temporal function timex3. The value is the identifier (tmx3id) of
  the timex3.

\item \texttt{mod} (optional): Used for temporal modifiers that cannot be
  expressed either within value proper, or via links or temporal
  functions. Possible values:
  \begin{itemize}
  \item BEFORE
  \item AFTER
  \item ON\_OR\_BEFORE
  \item ON\_OR\_AFTER
  \item LESS\_THAN
  \item MORE\_THAN
  \item EQUAL\_OR\_LESS
  \item EQUAL\_OR\_MORE
  \item START
  \item MID
  \item END
  \item APPROX
  \end{itemize}

\item \texttt{anchorTimeID} (optional): point to another timex3 in case of
  expression such as ``last week'', which have a functional
  interpretation. \texttt{anchorTimeID} provides the reference point to
  which the function interpretation applies. The value is the identifier
  (tmx3id) of the timex3.
\item \texttt{comment} (optional): optional comment.
\end{itemize}

The \texttt{<timex3>} element may have one sub-element, \texttt{<span>},
which groups the word forms the expression is referring to\footnote{Note
  that usually span elements refer to terms, but timex3 span elements refer
  to word forms.}.

\begin{Verbatim}[fontsize=\small]
<timeExpressions>
  <timex3 id="tmx1" type="DURATION">
    <span>
      <target id="w17"/>
      <target id="w18"/>
      <target id="w19"/>
      <target id="w20"/>
    </span>
  </timex3>
  <timex3 id="tmx2" type="DATE">
    <span>
      <target id="w22"/>
    </span>
  </timex3>
</timeExpressions>
\end{Verbatim}

\subsection{Temporal relations}
\label{sec:temporal-relations}

Temporal relations describe several relations that hold between events and
time expressions of a document. In NAF, all temporal relations are encoded
within the \texttt{temporalRelations} layer. The layer comprises elements of
type \texttt{tlink}, one per temporal relation, which mimics the semantics
of \texttt{tlink} elements of TimeML.

The \texttt{tlink} element is an empty element which contains the following
attributes:

\begin{itemize}
\item \texttt{id} (\textbf{required}): unique identifier starting with
  the prefix ``tlink'' (``tlink1'', ``tlink2'', etc.)
\item \texttt{from} (\textbf{required}): the source of the temporal relation.
\item \texttt{to} (\textbf{required}): the target of the temporal relation.
\item \texttt{fromType} (\texttt{required}): the type of the element the
  \texttt{from} attribute points to. It can be one of the following:
  \begin{itemize}
  \item event. The element is an event.
  \item timex. The element is a time expression.
  \end{itemize}
\item \texttt{toType} (\texttt{required}): the type of the element the
  \texttt{to} attribute points to. The possible values are the same as the
  \texttt{fromType} attribute.
\item \texttt{reltype} (\texttt{required}): the type of the
  relation. Usually it will have the same value as the \texttt{relType}
  attribute of TimeML's \texttt{tlink} element, namely:
  \begin{itemize}
  \item BEFORE
  \item AFTER
  \item INCLUDES
  \item IS\_INCLUDED
  \item DURING
  \item SIMULTANEOUS
  \item IAFTER
  \item IBEFORE
  \item IDENTITY
  \item BEGINS
  \item ENDS
  \item BEGUN\_BY
  \item ENDED\_BY
  \item DURING\_INV
  \end{itemize}
\end{itemize}

Here is an example of temporal relations in NAF:
\begin{Verbatim}
<temporalRelations>
  <tlink from="coevent46" fromType="event"
         relType="SIMULTANEOUS"
         to="coevent47" toType="event"/>
  <tlink from="tmx0" fromType="timex"
         relType="AFTER"
         to="tmx1" toType="timex"/>
  <tlink from="coevent48" fromType="event"
         relType="BEFORE"
         to="tmx0" toType="timex"/>
</temporalRelations>
\end{Verbatim}

The first \texttt{tlink} states that events \texttt{coevent46} and
\texttt{coevent47} occur simultaneously. The second states that time
expression \texttt{tmx0} occurred after time expression \texttt{tmx1}. The
last \texttt{tlink} states that event \texttt{coevent48} occurred before
time expression \texttt{tmx0}.

%%% Local Variables:
%%% mode: latex
%%% TeX-master: "naf"
%%% End:
